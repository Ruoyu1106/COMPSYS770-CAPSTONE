\documentclass{capstone}
\usepackage[T1]{fontenc}
\usepackage{parskip}
\usepackage{geometry}
\geometry{margin=2.5cm}
\usepackage{bookman}
\usepackage{unnumberedtotoc}
\usepackage{graphicx}
\usepackage[font={small,it}]{caption}

\begin{document}

\titlepage
	{\includegraphics[width=\paperwidth]{ecse-decal-title}}
	{
		\centering
		{\Large ECSE Capstone Final Report\par}
		\vspace{16pt} 
		{Team \#1\par} 
		\vspace{16pt}
		{Semester 1, 2024\par} 
	}

\tableofcontents
\newpage

\addsec{Executive Summary}
This section should be a one-page summary of your project: a busy reader should be able to read just this page and know all the high-level details of your project. It must be different from the introduction and conclusions.

Some suggestions of what to include:

\begin{itemize}
	\item Why is this project important?
	\item What value will it provide to Plunket?
	\item What are the key technical components?
	\item What is unique about your project?
	\item What are the next steps to bring this project to fruition?
\end{itemize}

\newpage

\addsec{Introduction}

Introduce the project, and what you plan to do. Focus on what value your project will add for Plunket. This section should be at a high level.

Remember: this is in response to a Request for Proposals. While you can include some background information on Plunket, you should remember that they released the Request for Proposals. Assume they know who they are and what they are asking for.

\addsec{Project Proposal}

Now that you have introduced the project, provide more detail on what you are actually going to implement. You can structure this section anyway you want, but at should cover the following points:

\begin{itemize}
    \item What unique functionality will \textbf{YOUR} project provide that will add value to Plunket?
    \item How have you considered privacy, regulatory obligations, te tiriti o Waitangi, and sustainability?
    \item How will your project meet the technical requirements from the brief?
\end{itemize}


Include some details on the next steps to implement your project.

\subsection*{Images and Tables}

Include images and tables to help the reader understand your project. 

\begin{figure}[h!]
    \centering
    \includegraphics{plunket-medium}
    \caption{The logo for Whānau Āwhina Plunket. You have permission to use this logo in your reports.}
    \label{fig:logo}
\end{figure}

Any tables and images you include must have a caption and be referenced in the text. For example, Figure~\ref{fig:logo} shows the logo for Plunket that you may use in your reports. Where possible, use the referencing functionality in \LaTeX{} (\textbackslash label and \textbackslash ref), so that the numbers are correct.

\subsection*{Page Limits}

Your final report must be 25 pages or less. Learning to write within a constrained space is an important skill for an engineer. You will have to decide what are the key points and information that is required for your proposal. 

You may want to change the font type, font size, margins, or other whitespace in the template to make more space and stay within the page limit. \textbf{DO NOT} change any of these! If you do, you will lose marks.

The page limit does not include the title page, table contents, executive summary, appendices, or any blank pages. You may include up to twelve additional pages for appendices.

\addsec{Conclusions}

Re-cap the main points from your project. You are trying to persuade the reader that your project is the best project, and that they should choose your project over any of the others. Do not repeat the introduction: this section should be significantly different.

\newpage
\addsec{Appendices}

You may include appendices if desired. However, your report should stand on its own (i.e., you cannot use the appendices to exceed the page limit of the main report.) 

Appendices must start on a new page.

\end{document}