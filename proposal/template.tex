\documentclass{capstone}
\usepackage[T1]{fontenc}
\usepackage{parskip}
\usepackage{geometry}
\geometry{margin=2.5cm}
\usepackage{bookman}
\usepackage{unnumberedtotoc}
\usepackage{graphicx}
\usepackage[font={small,it}]{caption}

\begin{document}

\titlepage
	{\includegraphics[width=\paperwidth]{ecse-decal-title}}
	{
		\centering
		{\Large ECSE Capstone Project Proposal\par}
		\vspace{16pt} 
		{Anonymous\par} 
		\vspace{16pt}
		{Semester 1, 2024\par} 
	}

\tableofcontents
\newpage

\addsec{Introduction}

Introduce the project, and what you plan to do. Focus on what value your project will add for Plunket. This section should be at a high level.

Remember: this is in response to a Request for Proposals. While you can include some background information on Plunket, you should remember that they released the Request for Proposals. Assume they know who they are and what they are asking for.

\addsec{Project Proposal}

Now that you have introduced the project, provide more detail on what you are actually going to implement. You can structure this section anyway you want, but at should cover the following points:

\begin{itemize}
    \item What unique functionality will \textbf{YOUR} project provide that will add value to Plunket?
    \item How have you considered privacy, regulatory obligations, te tiriti o Waitangi, and sustainability?
    \item How will your project meet the technical requirements from the brief?
\end{itemize}


Include some details on your project plan (how will your team implement the prototype within the timeframe?) Remember your report is anonymous: do not include any details that will identify team members (e.g., names, email addresses, etc.)

\subsection*{Images and Tables}

Include images and tables to help the reader understand your project. 

\begin{figure}[h!]
    \centering
    \includegraphics{plunket-medium}
    \caption{The logo for Whānau Āwhina Plunket. You have permission to use this logo in your reports.}
    \label{fig:logo}
\end{figure}

Any tables and images you include must have a caption and be referenced in the text. For example, Figure~\ref{fig:logo} shows the logo for Plunket that you may use in your reports. Where possible, use the referencing functionality in \LaTeX{} (\textbackslash label and \textbackslash ref), so that the numbers are correct.

\subsection*{Page Limits}

Your project proposal must be 12 pages or less. Learning to write within a constrained space is an important skill for an engineer. You will have to decide what are the key points and information that is required for your proposal.

You may want to change the font type, font size, margins, or other whitespace in the template to make more space and stay within the page limit. \textbf{DO NOT} change any of these! If you do, you will lose marks.

The page limit does not include the title page, table contents, feedback request, appendices, or any blank pages. You may include up to six additional pages for appendices.

\addsec{Conclusions}
Re-cap the main points from your proposal. These should include the key take away points for the reader. Do not repeat the introduction: this section should be significantly different.

\newpage
\addsec{Feedback}

An important part of the project proposal is receiving feedback. If there is any specific feedback you would like, you can include it as an additional page in the proposal. Remember, your proposal will be read by your peers and your mentors, so pitch your requests at their level.

This page will not marked. It must start on a new page. 

\newpage
\addsec{Appendices}
You may include appendices if desired. However, your report should stand on its own (i.e., you cannot use the appendices to exceed the page limit of the main report.) 

Appendices must start on a new page.

\end{document}